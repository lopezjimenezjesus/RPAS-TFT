
\documentclass[9t,twocolumn]{article}
\usepackage[left=1in,top=1in,right=1in,bottom=1in]{geometry}
\usepackage{booktabs}
\usepackage{rotating}
\usepackage{graphicx}
\usepackage{setspace}
\onehalfspacing
\usepackage{helvet}
\renewcommand{\familydefault}{\sfdefault}

\usepackage{caption}
\newcommand*{\authorfont}{\fontfamily{phv}\selectfont}
\usepackage[]{mathpazo}


  \usepackage[T1]{fontenc}
  \usepackage[utf8]{inputenc}
  \usepackage[nottoc,notlof,notlot]{tocbibind} 
  \renewcommand\refname{Referencias} % change original References title
  \renewcommand{\contentsname}{Tabla de contenidos  }



\usepackage{abstract}
\renewcommand{\abstractname}{}    % clear the title
\renewcommand{\absnamepos}{empty} % originally center

\renewenvironment{abstract}
 {{%
    \setlength{\leftmargin}{0mm}
    \setlength{\rightmargin}{\leftmargin}%
  }%
  \relax}
 {\endlist}

\makeatletter
\def\@maketitle{%
  \newpage
%  \null
%  \vskip 2em%
%  \begin{center}%
  \let \footnote \thanks
    {\pandoc_args: [
      "-V", "classoption=twocolumn"
    ]{18}{20}\selectfont\raggedright  \setlength{\parindent}{0pt} \@title \par}%
}
%\fi
\makeatother




\setcounter{secnumdepth}{0}



\title{Contribución de los RPAS en investigación y conservación en espacios
protegidos: presente y futuro  }



\author{\Large Autor: Jesús Jiménez López Tutora: Margarita Mulero-Pázmány\vspace{0.05in} \newline\normalsize\emph{Curso de Experto Universitario en Vehículos Aéreos no Tripulados y sus
Aplicaciones Civiles. Universidad de Cádiz}  }


\date{}

\usepackage{titlesec}

\titleformat*{\section}{\LARGE\bfseries}
\titleformat*{\subsection}{\normalsize\bfseries}
\titleformat*{\subsubsection}{\normalsize\itshape}
\titleformat*{\paragraph}{\normalsize\itshape}
\titleformat*{\subparagraph}{\normalsize\itshape}


\usepackage{natbib}
\bibliographystyle{apsr}


\newtheorem{hypothesis}{Hypothesis}
\usepackage{setspace}

\makeatletter
\@ifpackageloaded{hyperref}{}{%
\ifxetex
  \usepackage[setpagesize=false, % page size defined by xetex
              unicode=false, % unicode breaks when used with xetex
              xetex]{hyperref}
\else
  \usepackage[unicode=true]{hyperref}
\fi
}
\@ifpackageloaded{color}{
    \PassOptionsToPackage{usenames,dvipsnames}{color}
}{%
    \usepackage[usenames,dvipsnames]{color}
}
\makeatother
\hypersetup{breaklinks=true,
            bookmarks=true,
            pdfauthor={Autor: Jesús Jiménez López Tutora: Margarita Mulero-Pázmány (Curso de Experto Universitario en Vehículos Aéreos no Tripulados y sus
Aplicaciones Civiles. Universidad de Cádiz)},
             pdfkeywords = {RPAs, UAVs, drones, espacios naturales, conservación, biodiversidad,
investigación, innovación},  
            pdftitle={Contribución de los RPAS en investigación y conservación en espacios
protegidos: presente y futuro},
            colorlinks=true,
            citecolor=blue,
            urlcolor=blue,
            linkcolor=magenta,
            pdfborder={0 0 0}}

\urlstyle{same}  % don't use monospace font for urls



\begin{document}
	
% \pagenumbering{arabic}% resets `page` counter to 1 
%
% \maketitle

{% \usefont{T1}{pnc}{m}{n}
\setlength{\parindent}{0pt}
\thispagestyle{plain}
{\fontsize{10}{12}\selectfont\raggedright 
\maketitle  % title \par  

}

{
   \vskip 13.5pt\relax \normalsize\fontsize{10}{12} 
\textbf{\authorfont Autor: Jesús Jiménez López Tutora: Margarita Mulero-Pázmány} \hskip 15pt \vskip 8.5pt  \emph{\small Curso de Experto Universitario en Vehículos Aéreos no Tripulados y sus
Aplicaciones Civiles. Universidad de Cádiz}   

}

}






\begin{abstract}

    \hbox{\vrule height .2pt width 39.14pc}

    \vskip 8.5pt % \small 

\noindent En este estudio se procedió a identificar y resumir el estado actual y
las tendencias en el uso de los drones en proyectos científicos con
fines de conservación en espacios naturales protegidos, mediante la
recopilación y revisión de material bibliográfico en forma de artículos
científicos, revistas, proyectos de conservación y otras fuentes de
información relevantes.


\vskip 8.5pt \noindent \emph{Palabras claves}: RPAs, UAVs, drones, espacios naturales, conservación, biodiversidad,
investigación, innovación \par

    \hbox{\vrule height .2pt width 39.14pc}



\end{abstract}


{
\hypersetup{linkcolor=black}
\setcounter{tocdepth}{2}
\tableofcontents
}


\vskip 6.5pt

\noindent \doublespacing \section{Introducción}\label{introduccion}

Las aplicaciones de los vehículos aéreos no tripulados (RPAs, UAVs,
drones) en el campo de la conservación han sido planteadas en un número
cada vez mayor de artículos científicos. Durante los últimos años ha
habido un incremento significativo de las líneas de investigación sobre
vida silvestre que hacen uso de RPAS \citep{Linchant2015},
\citep{Christie2016}. Este auge ha conducido a un mayor desarrollo de
metodologías innovadoras de aplicación directa en espacios naturales
protegidos, con objeto de determinar su aplicabilidad frente a
instrumentos tradicionales de apoyo a la conservación, en sus diferentes
facetas. Si bien el concepto ha evolucionado a lo largo del tiempo,
actualmente los espacios naturales protegidos son aquellos en los que la
intervención del hombre no ha llegado a alterar de forma significativa
la presencia y funcionamiento de los elementos bióticos y abióticos que
lo integran \citep{Bravo2008}. Cumplen con finalidades de protección y
conservación del medio biofísico y cultural y donde se promueve
iniciativas en el ámbito científico y educativo, de restauración,
actividades recreativas y turísticas compatibles con el medio natural y
acciones de índole socioeconómica enmarcadas en el desarrollo
sustentable del territorio. Están amparados bajo algún instrumento
nacional o internacional de protección y regulados de forma acorde a
través de los planes de manejo específicos de la región.

A pesar de que el número de áreas protegidas ha experimentado un aumento
sensible a nivel mundial, con un 15.4 \% de la superficie terrestre y un
8.4\% de las áreas marinas bajo alguna figura de protección
\citep{juffe2014protected} , hay autores que resaltan la necesidad de
mejorar las herramientas de gestión para asegurar la efectividad de la
conservación de la biodiversidad en áreas protegidas \citep{Chape2005}.
Por otra parte algunas áreas protegidas sufren procesos de degradación,
continuan disminuyendo en tamaño o han dejado de existir
\citep{Mascia2011}, y en ocasiones han sido declaradas bajo criterios
oportunistas que no reflejan necesariamente el valor ecológico de los
ecosistemas a preservar \citep{Knight2007}. En un reciente informe
realizado por la Sociedad Zoológica de Londres \citep{Living2016} , se
calcula que el tamaño de las poblaciones de vida silvestre ha disminuido
en un 52 \% en el periodo de 1970 hasta 2012. Todos los indicios apuntan
al hombre como principal desencadenante de la ya denominada sexta
extinción masiva en nuestro planeta \citep{Barnosky2011}. Tanto es así,
que algunos investigadores comienzan a hablar del antropoceno, como
inicio de una nueva época en el periodo Cuaternario. La fragmentación
del habitat, el aumento de la contaminación, especialmente grave en
ecosistemas de agua dulce, la sobreexplotación de los recursos, las
consecuencias a nivel global del cambio climático y el impacto de las
especies invasoras sobre poblaciones autóctonas han sido identificados
como las principales amenazas para la diversidad biológica . El Grupo
sobre Observaciones de la Tierra (GEOBON) ha identificado un conjunto de
variables esenciales para la biodiversidad \citep{Pereira2013} con
objeto de desarrollar un abanico de indicadores que permitan conocer el
estado global de nuestros ecosistemas y ayuden a la mejor toma de
decisiones en materia de biodiversidad mediante la integración de
técnicas de observación remota y observaciones in-situ como piezas clave
para el levantamiento de información ambiental \citep{Forum2008}. Por
otra parte, el Convenio sobre la Diversidad Biológica desarrollado como
parte del Programa de las Naciones Unidas para el Medio Ambiente (PNUMA)
estableció en Nagoya (Japón) un plan estratégico para el periodo
2011-2020 que incluye las metas de Aichi para la diversidad biológica.
Dentro de los objetivos planteados cobra especial relevancia en el
contexto actual el aumento de los sistemas de áreas protegidas de
especial importancia para la biodiversidad y los servicios ecosistémicos
(Meta 11) y se establece una serie de criterios de gobernanza, equidad,
gestión, representatividad y conectividad ecológica para la inclusión de
áreas prioritarias para la conservación. Para hacer frente a una
situación cada vez más insostenible, es necesario desarrollar soluciones
noveles que mejoren nuestra compresión de los ecosistemas y permitan
tomar medidas encaminadas a la preservación de la biodiversidad. En este
contexto, el presente estudio realiza una revisión pormenorizada del
desempeño de los RPAS en materia de conservación y gestión de áreas
protegidas, en la medida que se superen las barreras técnicas y legales
que limitan su efectividad.

Existen actualmente algunas iniciativas que tratan de recoger el estado
actual de los RPAS en las áreas de la ecología y conservación. Con fecha
reciente de finalización , la revista \emph{Remote Sensing in Ecology
and Conservation} hizo una llamada a la comunidad científica para el
envío de propuestas afines, con objeto de sensibilizar a estudiantes y
profesionales y demostrar el uso responsable de RPAS. Es de esperar que
del resultado de este llamamiento se produzca un aumento significativo
de la literatura científica en este ámbito. Por otro lado, es remarcable
la mayor presencia de portales en internet que centran su actividad en
torno a las aplicaciones con RPAS. En el campo de la investigación
aplicada en conservación \url{http://conservationdrones.org/} es uno de
los sitios de referencia. Sus objetivos se enmarcan en la facilitación
del uso y desarrollo de RPAs en actividades conservacionistas. En su web
se pueden consultar casos de usos de RPAS cuyos resultados, dado el
caracter pionero de estos estudios, no siempre aparecen reflejados en
artículos científicos. Dentro de las aplicaciones de los RPAS con
caracter general destaca la comunidad online
\url{http://diydrones.com/}, en la que tiene gran acogida el uso de
plataformas abiertas, de gran popularidad frente a los tradicionales
sistemas cerrados promovidos por compañias comerciales del sector. Esto
ha dado como resultado la reducción de los costes de estos equipos,
junto con el software asociado, permitiendo acercar la tecnología
disponible a un mayor número de usuarios y organizaciones. Estas
plataformas abiertas tienen la ventaja adicional de tener un mayor grado
de personalización de los equipos. El incremento en la flexibilidad en
el montaje de diferentes sensores y sistemas de control permite atender
a las necesidades específicas de cada proyecto, dentro del propio grupo
de investigación \citep{Koh2012}.

A partir de las fuentes consultadas ha sido posible determinar el estado
actual de los RPAS y las tendencias futuras en contraposición a las
técnicas tradicionales de apoyo a la conservación en espacios naturales
protegidos, tanto en su vertiente científica como conservacionista. En
este sentido, las limitaciones desde el punto de vista financiero y
tecnológico de la teledetección, por la cual se obtienen imágenes de la
superficie terrestre a partir de sensores instalados en plataformas
aéreas o espaciales, son descritas por diversos autores \citep{Koh2012}.
Si bien es posible adquirir imágenes satelitales a bajo coste o
prácticamente nulo (LandSat, MODIS, Sentinel, etc.), la mayor parte de
estas plataformas operan a escala global o regional. La limitada
resolución espacial y temporal, junto con los problemas de presencia de
nubes especialmente acusados en zonas tropicales, reduce la viabilidad
de la teledetección en la recolección de datos a escala suficientemente
detallada para hacer frente a los requerimientos de estudios ecológicos
a nivel de hábitat, especies o poblaciones \citep{Wulder2004}. Además,
el gran tamaño de estas áreas protegidas reducen en muchos casos la
efectividad y aumenta significativamente los costes de los trabajos de
campo, mientras que aumenta los riesgos en zonas especialmente
inaccesibles, por lo que los RPAS se han posicionado como un complemento
adecuado para las actividades de conservación \citep{Zahawi2015}. En
paises en vías de desarrollo, especialmente sensibles en cuanto a
dotaciones presupuestarias, se han desarrollado con gran éxito programas
de monitoreo y vigilancia a partir del uso del RPAS, eliminando los
inconvenientes descritos con anterioridad. Los vehículos aéreos
tripulados ofrecen en principio una mejor alternativa en la captura
imágenes de la superficie terrestre, sin embargo su uso no está
justificado en estudios a escala local, debido a costes operacionales
excesivamente altos. Por otro lado, el riesgo de sufrir accidentes
aéreos es mayor, situándose como primera causa de mortandad en
especialistas en vida silvestre en los Estados Unidos \citep{Sasse2003}.

Con objeto de reducir el impacto de los drones en estudios de fauna ,
algunos experimentos analizan la respuesta de aves frente a RPAS
\citep{Vas2015}. Otros ensayos se centran en mamíferos y miden el estreś
fisiológico y posibles cambios en la etología frente a vuelos realizados
con RPAS \citep{Ditmer2015}. Fruto de los resultados obtenidos, se están
comenzando a documentar manuales de buenas prácticas y recomendaciones
con objeto de reducir el impacto negativo en el bienestar y evitar
perturbaciones en los patrones de comportamiento de las especies.

Finalmente, algunos autores señalan la necesidad de mejorar el marco
regulatorio respecto al uso civil de los RPAS \citep{Nugraha2016}. En
los Estados Unidos y en la mayoría de los paises de Europa consultados,
se han adoptado leyes provisionales que en cierta medida equiparan el
manejo de los RPAS con el de aeronaves tradicionales. Este tipo de
restricciones podría limitar las posibilidades de uso de los RPAS en el
ámbito de la conservación, por lo que se hace patente la necesidad
urgente de armonizar la legislación en relación a este tipo de
actividades. En términos generales, la situación en America Latina es
desigual, con algunos paises que siguen sin desarrollar leyes
específicas para hacer frente al auge de los RPAS tanto en el sector
civil como militar \citep{Nacion2013}. Africa es uno de los continentes
donde el impacto de los drones en conservación ha tenido mayor
repercusión. Sin embargo, según la opinión de algunos conservacionistas,
su uso no ha estado exento de problemas, dando como resultado gobiernos
que han prohibido total o parcialmente la operación con drones,
argumentando problemas de seguridad nacional en detrimento de la
protección de los espacios naturales protegidos \citep{Andrews2014}.

La incertidumbre de los usuarios ha promovido el desarrollo de
asociaciones con objeto de asesorar sobre los aspectos legales a tener
en cuenta durante la operación. En España, la Asociación Española de
Drones y Afines \url{https://www.aedron.com} promueve un uso consciente
y responsable de los RPAS y organiza seminarios para informar a los
socios sobre temas de interés. En su web se puede consultar el borrador
de la nueva normativa que regula la utilización civil de las aeronaves
pilotadas por control remoto en España \citep{Aedron2017}. A nivel
global han surgido otras iniciativas, siendo la Asociación Internacional
para Sistemas de Vehículos No Tripulados (AUVSI)
\url{http://www.auvsi.org} la organización sin fines de lucro más grande
del mundo dedicada al avance de la comunidad de usuarios de sistemas
aéreos no tripulados.

\section{Métodos}\label{metodos}

Para alcanzar los objetivos propuestos se procedió a la revisión
bibliográfica de artículos, literatura gris, tesis de postgrado, sitios
web y revistas especializadas, siguiendo una línea similar a otros
estudios realizados con anterioridad \citet{Linchant2015},
\citep{Christie2016} Mediante artículos seleccionados para el curso de
Experto Universitario en Vehículos Aéreos no Tripulados y sus
Aplicaciones Civiles organizado por la Universidad de Cádiz en su
edición de 2016-2017, junto con herramientas como Google Schoolar,
ResearchGate y Mendeley Desktop se obtuvo la mayor parte de la
bibliografía seleccionada, mientras que el uso de los motores de
búsqueda por internet incluyeron el resto de materiales mencionados. Los
principales criterios de búsqueda por palabras claves incluyeron los
vehículos aéreos no tripulados en sus diversas acepciones y acrónimos
(RPAS, UAV, drones, etc.), junto con una variedad de términos que hacen
referencia a áreas naturales protegidas, fundamentalmente en inglés.
Dicha actividad tuvo lugar hasta el mes de Abril, 2017.

La información recolectada se categorizó según el propósito de
aplicación de los RPAS en relación directa o indirecta con la
conservación en espacios protegidos. La mayoría de las fuentes
analizadas se centran en proyectos de conservación a escala local y
estudios de viabilidad de los RPAS en la caracterización de poblaciones
y comunidades de vida silvestre, especialmente en estudios de
distribución y abundancia. La literatura comienza a ser igualmente
prolífica en actividades de monitoreo y mapeo en ecosistemas terrestres
y acuáticos. De igual manera se observa una tendencia cada vez mayor de
artículos dedicados al uso de RPAS en el control y vigilancia de áreas
protegidas. Adicionalmente se revisan algunos aspectos de índole social
recogidos en los materiales seleccionados y que son motivo de
controversia, con especial referencia a la privacidad de las personas y
el bienestar de las especies estudiadas, las implicaciones éticas y
legales y la repercusión en la efectividad de los RPAS en la
conservación a largo plazo. En cualquier caso, dado el carácter
multidisciplinar y multipropósito de estos estudios existe cierto solape
entre los objetivos marcados dentro de cada proyecto, por lo que se ha
tenido en cuenta aquellos objetivos que mayor peso tienen en el contexto
de la investigación.

La información seleccionada se presenta en formato tabular,
identificando los paises implicados, el propósito principal de cada
estudio, junto con las técnicas y materiales empleados, haciendo
referencia explicita al tipo de aplicación y plataformas de vuelo, tanto
de ala fija como de pala rotatoria. Finalmente se discuten los posibles
escenarios que presentan los RPAS como herramientas fundamentales para
contribuir a la consecución de los planes de conservación en espacios
protegidos, destacando algunas tendencias y oportunidades que aún no han
sido convenientemente explotadas.

\begin{sidewaystable}
\centering
\captionsetup{font=scriptsize,labelfont=scriptsize}
\caption{Estudios con RPAS realizados en Areas protegidas, caracteristicas tecnicas de la plataforma y especies objetivos}
\label{my-label}
\tiny
\begin{tabular}{p{2.5cm}p{1cm}p{3cm}p{1cm}p{2cm}p{2cm}p{1cm}p{2cm}p{2cm}p{1cm}p{0.5cm}}
\cmidrule(r){1-11}

Estudio & Tipo de Estudio & Objetivo/s & País & Lugar & Especie & Tipo RPAS & Modelo RPAS & Sensor & Georref. & Costo \\ \cmidrule(r){1-11}

\ ESTUDIOS DE FAUNA Y VIDA SILVESTRE \\ 


\cite{PazmanyMulero2015}  & Si & Estudio comparativo modelos distribución de especies & España & Parque Nacional de Doñana & Bos taurus  & Ala fija & Easy Fly plane, Ikarus autopilot, Eagletree GPS logger & Panasonic Lumix LX-3 11MP & Si & 
5700 euros \\ 

\citealt{Hodgson2013} & Si & Determinar la eficacia para detectar e identificar dugongs.  Comprobar la actitud de los RPAS en diferentes condiciones ambientales. Determinar la resolución ideal  & Australia & Shark Bay Marine Park & Dugong & Ala fija &  ScanEagle & Nikon® D90 12 megapixel digital SLR camera  & Si & ?  \\ 


\cite{Wilson2017}  & No & Monitoreo bioacústico con RPAS & USA & State Game Lands & Pájaros  & Multicóptero & DJI Phantom 2 & ZOOM H1 Handy Recorder  & Si & ? \\ 

\cite{Szantoi2017}  & Si & Mapeo de hábitat & Indonesia & Gunung Leuser National Park & Pongo abelii (orangután)  & Ala fija & Skywalker & Canon S100  & Si & \$ 4000 \\ 


\cite{Bayram2016}  &  No & Detección de collares VHF & ? & & Osos  & Multirotor & DJI F550 hexarotor, Pixhawk autopilot & Telonics MOD-500 VHF, Uniden handheld scanner  & Si & ? \\ 

\cite{Christie2016}  &  Si  & Estimación abundancia & USA &  Aleutian Islands & Eumetopias jubatus (Steller sea lion)  & Multirotor & APH- 22 hexacopter & ?  & Si & \$ 25.000 , \$ 3000 vessel support, or \$ 1700 per site \\ 

\cite{Christie2016}  &  Si & Estimación abundancia & USA &  Monte Vista National Wildlife Refuge
 & Grus canadensis (sandhill cranes)  & Ala fija & Raven RQ- 11A & ?  & Si & \$ 400 \\ 

\end{tabular}
\end{sidewaystable}

\begin{sidewaystable}
\centering
\captionsetup{font=scriptsize,labelfont=scriptsize}
\caption{Monitoreo de ecosistemas terrestres y acuáticos}
\label{my-label}
\tiny
\begin{tabular}{p{2.5cm}p{1cm}p{3cm}p{1cm}p{2cm}p{2cm}p{1cm}p{2cm}p{2cm}p{1cm}p{0.5cm}}
\cmidrule(r){1-11}

Estudio & Área protegida & Objetivo/s & País & Lugar & Especie & Tipo RPAS & Modelo RPAS & Sensor & Georref. & Costo \\ \cmidrule(r){1-11}

  \ MONITOREO DE ECOSISTEMAS TERRESTRES Y ACUÁTICOS \\ 
  
  \cite{Perroy2017}  & No & Monitoreo de plantas invasoras & USA & Pahoa, Hawai & Miconia calvescens & Multirotor & DJ Inspire-1 & DJI FC350 camera  & Si & ?  \\ 
  
  \cite{Ivosevic2015}  & Si & Monitoreo de habitats en zonas restringidas; Modelos; Comprobar la actitud de los RPAS en diferentes condiciones ambientales. & South Korea & Chiaksan National Park;Taeanhaean National Park &  Especie & Multicóptero & DJI Phantom 2 Vision+ , built-in full HD videos  1080p/30fps and 720p/60fps, 14 megapixels 4384x3288 resolution camera & Si & Costo \\ 
  
  
  \cite{Lisein2015}  & No & Discriminación de especies de  hoja caduca, inventario forestal & Bélgica & Grand-Leez & English oak, birches (Betula pendula Roth. and Betula pubescens Ehrh.), sycamore maple (Acer pseudoplatanus L.), common ash (Fraxinus excelsior L.) and poplars (two distinct varieties of cultivated Populus spp.) & Ala fija & Gatewing X100  & Ricoh GR2 GR3 GR4 10 megapixels CCD  & Si & ?  \\ 
  
  \cite{Puttock2015}  & Si & Caracterización ecosistemas afectados por la actividad del castor & UK & Devon Beaver Project site & Eurasian beaver (Castor fiber) & Multirotor & 3D Robotics Y6 & Canon ELPH 520 HS  & Si & ?  \\ 
  
  \cite{Zahawi2015}  & No & Caracterización estructura bosques tropicales para acciones de restauración & Costa Rica & Devon Beaver Project site & Varias especies & Multirotor & 3D Robotics Y6 & Canon S100  & Si & \$ 1500 \\ 
  
  
  \cite{Bustamante2015}  & Si &  Monitoreo de bosques & Brasil & Riverine Forests (Permanent Protected Areas), Rio de Janeiro, Barrãcao do Mendes, Santa Cruz and São Lorenço & DJI Phantom Vision 2S  & RGB digital camera with 14 mega pixels & Riverine Areas & Si & \$ 9700  \\ 
  
  \cite{Gini2012}  & Si & Modelamiento 3D y clasificación de especies arbóreas & Italy & Parco Adda Nord
    & Multirotor  &  MicrodronesTM MD4-200 & RGB CCD 12 megapixels Pentax Optio A40, modified NIR Sigma DP1 with a Foveon X3 sensor & Varias especies & Si & ?  \\

\\ \cmidrule(r){1-11}



\end{tabular}
\end{sidewaystable}

\begin{sidewaystable}
  \centering
  \captionsetup{font=scriptsize,labelfont=scriptsize}
  \caption{Estudios con RPAS realizados en Areas protegidas, caracteristicas tecnicas de la plataforma y especies objetivos}
  \label{my-label}
  \tiny
  \begin{tabular}{p{2.5cm}p{1cm}p{3cm}p{1cm}p{2cm}p{2cm}p{1cm}p{2cm}p{2cm}p{1cm}p{0.5cm}}
  \cmidrule(r){1-11}
  
  Estudio & Área protegida & Objetivo/s & País & Lugar & Especie & Tipo RPAS & Modelo RPAS & Sensor & Georref. & Costo \\ \cmidrule(r){1-11}
  
  
  \ EVALUACIÓN DE INFRAESTRUCTURAS Y RIESGO \\
  
  \cite{Lobermeier2015} & No  & Mitigar el riesgo de colisión mediante la instalación de marcadores en líneas electríca & USA & ? & Especie& Aves  & Multirotor  & Mikrokopter Hexa XL  & KX 171 Microcam & ?  \\ 
  
  \cite{Mulero-Pazmany2014a} & Si  & Evaluación del riesgo riesgo eléctrico en nidos instalados en postes de alta tensión & España & Parque Nacional de Doñana &  Aves  & Ala fija  & Easy fly St-330 & GoPro Hero 2 11 MP, Panasonic LX3 11MP & Si & 7800 euros  \\ 
  
  
  \ VIGILANCIA \\ 
  
  \cite{Mulero-Pazmany2014}  & Vigilancia en áreas protegidas & Si & Africa & KwaZulu-Nata & black rhinoceros
  (Diceros bicornis), white rhinoceros (Ceratotherium simum)  & Ala fija  & Easy Fly St-330 & Panasonic Lumix LX-3 11 MP, GoPro Hero2, Thermoteknix Micro CAM microbolometer & Si & 13750 euros  \\ 

  
  \ ECOTURISMO  \\
  
  \cite{Hansen2016} & Si  & Monitoreo actividad visitantes  & Sweeden & Kosterhavet National Park &  Humanos  & ?  & ? & ?  & ? & ?  \\ 
  \cite{King2014} & Si  & Aplicaciones RPAs en actividades ecoturismo   & Sweeden & Kosterhavet National Park &  Humanos  & ?  & ? & ?  & ? & ?  \\ 
  
  
  \ ETICA ANIMAL  \\
  \cite{Vas2015} & Si  & Impacto de los RPAS en especies de aves lacustres  & Sweeden & Kosterhavet National Park &  Humanos  & ?  & ? & ?  & ? & ?  \\ 
  \cite{Ditmer2015} & Si  & Impacto de los RPAS en especies de aves lacustres  & Sweeden & Kosterhavet National Park &  Humanos  & ?  & ? & ?  & ? & ?  \\ 

  
  \end{tabular}
  \end{sidewaystable}

\section{Discusión}\label{discusion}

En discusión comenta el resultado de esa tabla y los porqués (ej se usan
más multicópteros que fixed por\ldots{}) y las limitaciones que señalan
los usuarios o conflictos que hayan podido encontrar (con el parque,
técnicos etc).

\subsection{Estudios de fauna y vida
silvestre}\label{estudios-de-fauna-y-vida-silvestre}

Uno de los temas centrales de la ecología es el desarrollo de modelos
geoestadísticos de distribución de especies que permiten inferir el
hábitat potencial o idóneo de los organismos a partir de la recolección
de información ambiental y datos de presencia procedentes de diversas
fuentes \citep{Mateo2011}. La radiotelemetría es uno de los métodos más
comunes para la recolección de datos de movimiento en individuos
marcados con geolocalizadores. Algunos estudios comparan el uso de RPAS
frente a estos sistemas \citep{PazmanyMulero2015},
\citep{Mulero-Pazmany2015} en animales de gran tamaño y fácilmente
identificables mediante imágenes aéreas de alta resolución, obteniendo
resultados similares en cuanto al rendimiento de los modelos pero con
diferencias notables en cuanto a costes derivados de la compra de los
equipos y gastos de logística, favoreciendo en este caso a los RPAS. Las
limitaciones financieras también afectan al tamaño del muestreo con el
uso de geolocalizadores, con el riesgo añadido de marcar individuos bajo
criterios no aleatorios, si bien se remarca la ventaja de estos sistemas
en cuanto a su capacidad para generar grandes volumenes de datos en un
periodo de tiempo mayor. En cuanto a la exactitud posicional, los RPAS
tienen un error máximo entre 1 y 3 metros, mientras que los errores del
GPS pueden ser mayores a 20 metros. En cualquier caso los autores
remarcan que ambas metodologías tienen potencial para complementarse a
lo largo de todas las fases del estudio. Otras técnicas innovadoras han
sido recientemente ilustradas en artículos cientificos que evaluan la
viabilidad del uso combinado de radiolocalizadores en RPAS en la
búsqueda de individuos marcados con radio collares VHF
\citep{Korner2010}, \citep{Bayram2016}, \citep{Cliff2015},
\citep{Leonardo2013}.

En determinados casos, frente a las dificultades para detectar
directamente a la especie de interés, los estudios se enfocan en la
localización y caracterización de sus áreas de cría y nidificación
\citep{VanAndel2015}. En áreas protegidas de gran extensión se han
ensayado con éxito el conteo de grandes mamíferos terrestres , no
habiéndose registrado reacciones adversas en vuelos realizados a cierta
altura \citep{Vermeulen2013}. La estimación de poblaciones de mamíferos
en ecosistemas marinos también ha sido documentado con resultados
positivos \citep{Hodgson2013}, mientras que en el apartado de aves se
han usado para estudiar las dinámicas poblacionales en colonias
\citep{Sarda-Palomera2012}. La utilidad de estos sistemas también queda
manifiesta en la inspección y caracterización de nidos de aves en zonas
de dificil acceso \citep{Weissensteiner2015}, permitiendo evaluar el
estado en el que se encuentran de forma menos intrusiva.

Dada la masiva cantidad de información que generada, no es de extrañar
que se hayan aplicado métodos desarrollados en el campo de la visión
computerizada, dirigidos al conteo automático de individuos capturados
en las escenas adquiridas por los sensores fotográficos
\citep{Lhoest2015},\citep{Abd-Elrahman2005a}, \citep{VanGemert2015}.
Esto conlleva una reducción de los costes respecto al conteo manual de
las escenas adquiridas, con la ventaja adicional de no estar sujetos en
mayor o menor medida a la interpretación del especialista. En este
sentido, los métodos de observación directa desde vehículos aéreos
tripulados también representan desventajas con respecto a la toma de
imágenes aéreas, puesto que necesitan un mayor número de observadores
que garantizen un conteo exahustivo de las poblaciones para evitar
errores en la estimación.

Fuera de la literatura científica, existen proyectos para el monitoreo
de la fauna tanto en ecosistemas marinos como terrestres. A partir de la
información recopilada en la comunidad online
\url{https://conservationdrones.org} se han identificado varios estudios
relacionados con el registro de individuos en poblaciones situadas en
áreas protegidas o frecuentemente visitadas por especies sujetas a
alguna figura de amenaza, siendo la mayoría de estos proyectos
respaldados por organizaciones no gubernamentales y centros de
investigación. Por ejemplo, un estudio realizado en la cuenca del
Amazonas en Brasil está experimentando el uso de drones para mejorar la
estimación de la densidad y abundancia de diferentes especies de
delfines, comparándolo con la observación directa realizada por
especialistas \citep{WichS2017}. Dentro de los objetivos de la
investigación se contempla la validación y armonización de ambas
metodologías y de forma indirecta, evaluar la viabilidad para su
aplicación regular en proyectos de monitoreo con similar propósito,
teniendo en cuenta el coste-beneficio de la ejecución.

\subsection{Evaluación de infraestructuras y
riesgo}\label{evaluacion-de-infraestructuras-y-riesgo}

Otros trabajos resaltan la utilidad de los RPAS en la evaluación del
riesgo de infraestructuras humanas y la puesta en marcha de medidas
preventivas frente a especies de aves que nidifican en postes de líneas
eléctricas de alta tensión, haciéndolas especialmente vulnerables a
colisiones y electrocutamiento. Para la ejecución de trabajos de
precisión donde la estabilidad, maniobrabilidad y el detalle en la
identificación es esencial \citep{Lobermeier2015} el uso de multirotores
es recomendado, mientras que en evaluaciones de estructuras lineales de
gran extensión en los que el costo y tiempo de vuelo es determinante en
contraposición a la resolución espacial, los vehículos de ala fija
ofrecen mejores ventajas \citep{Mulero-Pazmany2014a}, \citep{Zhang2016}.

Si bien estos estudios no están dirigidos exclusivamente a áreas
protegidas, podrían resultan de especial interés en zonas aledañas de
amortiguamiento, donde el desarrollo antrópico puede generar situaciones
de conflicto con la fauna circundante. Por ejemplo, se sabe que hay
ciertas especies de aves que nidifican en el suelo, especialmente en
zonas de cultivo de cereal. Como actividad previa a la cosecha,
realizada generalmente bajo procedimientos mecánicos, se podría realizar
un sobrevuelo para identificar posibles nidos, y en su caso, tomar las
medidas adecuadas para evitar su destrucción.

\subsection{Monitoreo y mapeo de ecosistemas terrestres y
acuáticos}\label{monitoreo-y-mapeo-de-ecosistemas-terrestres-y-acuaticos}

Durante las últimas decadas el auge de los sensores remotos a bordo de
plataformas aéreas o espaciales ha desencadenado un aumento de las
aplicaciones para el estudio de los ecosistemas \citep{Wulder2004}. Los
datos obtenidos han permitido desarrollar mapas de cobertura vegetal y
suelos, caracterizar hábitats, mejorar la compresión de la estructura y
función de las masas forestales, desarrollar modelos digitales de
elevaciones o levantar cartas geomorfológicas de aplicación en el
modelamiento de distribución de especies. El advenimiento de los RPAS ha
propiciado el análisis cuantitativo de hábitats a un nivel de detalle
que no ha sido posible hasta ahora, bien por motivos económicos o por
limitaciones propias de la ingeniería. Este impulso ha sido
especialmente notable con el desarrollo paralelo de sensores
multiespectrales e hiperespectrales adaptados a aeronaves de pequeño
tamaño, cuyo precio se espera disminuya según las tendencias del sector
tecnológico . Dentro de las actividades del Servicio Geológico de los
Estados Unidos (USGS) se han operado vuelos con objeto de clasificar la
cobertura vegetal en humedales \citep{USGS2014}. Otros estudios
monitorean la distribución de especies invasoras bajo diferentes
condiciones de vuelo y cobertura vegetal \citep{Perroy2017}, mientras
que la caracterización de masas forestales constituye un importante
apartado dado el número de artículos que abordan el problema desde
diferentes perspectivas. \citep{Gini2012} emplea un modelo de
cuadrocóptero y cámaras en RGB y NIR a baja altura en áreas de pequeña
extensión. Debido a la reducida fiabilidad y autonomía de la plataforma
y las dificultades para aumentar la capacidad de carga, la planificación
del vuelo se ve reducida a tres pasadas con un porcentaje del 80\% y
30\% de solape longitudinal y transversal
respectivamente.\citep{Lisein2015} realiza un análisis multitemporal de
la respuesta espectral frente a variaciones en la fenología en
diferentes especies de árboles de hoja caduca y concluye que la
variación espectral intraespecífica es de máximo interés para la
optimización de los algoritmos de clasificación y discriminación entre
especies. En su investigación, opera un modelo de RPAS de ala fija,
utiliza diferentes sensores en el rango visible e infrarrojo cercano y
optimiza los parámetros de vuelo con objeto de cubrir la máxima
superficie con el menor número de vuelos posible. \citep{Zahawi2015}
aplica la metodología \textbf{Ecosynth}, un conjunto de herramientas
para cartografiar y medir la vegetación en 3D utilizando cámaras
digitales y software de visión artificial de código abierto, con objeto
de evaluar la eficacia de las acciones de restauración en bosques
tropicales mediante RPAS, como alternativa viable para las medidas de
campo tradicionales y aplica diferentes modelos predictivos de presencia
de pájaros frugívoros a partir de los datos de altura y estructura del
dosel forestal.

\subsection{Vigilancia y apoyo para el cumplimiento de las leyes en
áreas
protegidas}\label{vigilancia-y-apoyo-para-el-cumplimiento-de-las-leyes-en-areas-protegidas}

Los RPAS también tienen especial proyección en el control y vigilancia
de áreas protegidas. Así lo demuestran diferentes experiencias enfocadas
principalmente en el control de la caza y pesca furtiva. Este tipo de
estudios se caracteriza por dar una mayor importancia a la mejora de los
sistemas de visión en primera persona (FPV) con objeto de obtener una
panorámica en tiempo real de la zona monitoreada, el uso de RPAS de ala
fija cuya mayor autonomía frente a los multirrotores permite cubrir una
mayor extensión, la necesidad de utilizar cámaras térmicas en
condiciones de baja visibilidad, usualmente relacionadas con horas de
mayor actividad furtiva, junto con avances en los sistemas de visión
computerizada programados para detectar la presencia de humanos y
especies sometidas a la presión de comercio ilegal en áreas protegidas.
\citep{Mulero-Pazmany2014} se enfocan en el rinoceronte africano y
constatan las ventajas del video en tiempo real frente a la toma de
fotografías, que necesitan un mayor tiempo de postprocesamiento.
Adicionalmente recalcan la necesidad de mejorar la resolución de las
sensores térmicos para aumentar las posibilidades de detectar
actividades sospechosas en horas nocturnas. Otras consideraciones
incluyen la falta de una regulación específica para actividades con RPAS
que permitan operar más allá del campo visual, el incoveniente de la
escala de trabajo en grandes extensiones de terreno, las condiciones
atmosféricas adversas que afectan a la capacidad de volar de los RPAS y
los posibles efectos negativos sobre la fauna, como ejemplos de alguno
de los retos que van a determinar la capacidad de integrar los RPAS en
actividades de control y vigilancia. {[}@{]} analiza las consecuencias
de la militarización de las prácticas de conservación, como tendencia
cada vez mayor en áreas naturales protegidas de todo el mundo e ilustra
el uso de RPAS a través de varios ejemplos. Respecto a zonas costeras,
una búsqueda rápida por internet permite recoger diversas iniciativas
que tratan de optimizar las labores de control de la pesca furtiva
mediante RPAS. Sin embargo no hemos podido constatar estudios
científicos que avalen tales iniciativas, por lo que se abre una vía
interesante para su investigación. Por ilustrar algunos de los numerosos
ejemplos, en Belize se realizó un estudio pionero para el monitoreo de
pesquerías mediante un modelo de ala fija Skywalker. El Gobierno de
Canarias está considerando el uso de RPAS en labores de control e
inspección en zonas de dificil acceso para hacer frente al furtivismo
\citep{Canarias2017} . Finalmente \url{http://soarocean.org/} es una
iniciativa de \textbf{National Geographic} y \textbf{Lindblad
Expedition} para el uso de drones de bajo coste en la protección de los
océanos.

\subsection{Ecoturismo}\label{ecoturismo}

El amplio abanico de posibilidades que ofrece la aplicación de los RPAS
en la industria ecoturística queda resumido en un artículo reciente, en
el que se exponen posibles actividades recreativas, oportunidades de
negocio, operaciones de búsqueda y rescate, mapeo y fórmulas para la
concesión de operaciones con RPAS en áreas designadas para tal uso.
Dentro de la aún escasa literatura, \citep{Hansen2016} valora la
eficacia de los RPAS en el monitoreo de visitantes en áreas marinas y
costeras, en combinación con otras soluciones innovadoras. Según el
autor los RPAS permitirían teóricamente operar bajo diferentes
condiciones ambientales, mejorando el nivel de detalle y ofreciendo una
cobertura continua en el flujo y comportamiento de los visitantes , en
contraposición a otras técnicas de uso habitual como la observación
manual o la instalación de redes de cámaras de vigilancia. La
implementación de este tipo de soluciones estaría sujeta a ciertas
consideraciones. Por ejemplo, aún no está claro si el ruido o la
presencia notable de los RPAS en áreas naturales podría afectar
negativamente la experiencia del turista o pertubar significativamente a
la fauna. Para que la industria ecoturística pudiera beneficiarse de los
RPAS, sería necesario mejorar nuestro conocimiento sobre estas
cuestiones planteadas y abordar las implicaciones éticas y legales
derivadas de su uso.

\subsection{Impacto de los RPAS en la fauna
silvestre}\label{impacto-de-los-rpas-en-la-fauna-silvestre}

La priorización del bienestar del animal debe ser incluida en las
aplicaciones de RPAS en el ámbito de la conservación. \citep{Vas2015}
obtienen resultados positivos que permiten aplicar los RPAS en estudios
de ornitología. En su estudio miden el impacto del color, la velocidad y
el ángulo de vuelo en las respuestas de comportamiento de aves lacustres
frente a la aproximación de multirrotores, siendo este último factor el
principal desencadenante en los patrones de comportamiento,
especialmente en aproximaciones desde la vertical, a un ángulo de 90º.
Finalmente concluyen con una serie de recomendaciones básicas y
consideran recomendable extender los ensayos a una amplia gama de drones
y especies.

\citep{Ditmer2015}

\section{Resultados y conclusiones}\label{resultados-y-conclusiones}

Todavía no esta redactado

A pesar de las dificultades para encontrar casos de uso de drones
especificamente realizados en áreas protegidas, la mayoría de los
estudios analizados tiene aplicacion directa en estas areas, puesto que
se supone incluyen la mayor concentración de la biodiversidad de
especies. Por otro lado

Una de las mayores limitaciones en el desarrollo y aplicación de los
RPAS en estudios de conservación se debe a las restricciones impuestas
por la legislación actual. Diversos autores mencionan los problemas
burocráticos para obtener permisos de investigación que hagan uso de
esta tecnología emergente \citep{FEE:FEE201513274}. Consideramos que una
regulación favorable permitiría aumentar las oportunidades en el sector,
estimulando la innovación técnológica.

Los conflictos sociales también ocupan un capítulo importante en el
futuro de los drones en conservación. Comunidades implicadas

Hasta ahora no ha habido un desarrollo especifico de drones que

En cuanto a ética animal, pilotos experimentados y conscientes.

En la mayoría de los estudios analizados se remarca el bajo coste
operacional de los drones frente a otras herramientas de conservación.
En estudios de caracter multitemporal o con necesidades de alta
resolución espacial las ventajas son especialmente patentes. Hay drones
que permiten seguir el objetivo\ldots{} limitación batería

Riesgos de los vehiculos aereos tripulados, costes,

Explicar que estudios usan multirotores y ala fija Mejoras en la
autonomia de vuelo, junto con el desarrollo de sensores de mayor
resolución. En al medida en la que aumente el desarrollo de sensores con
mayor resolución espectral, el abanico de posibilidades de

Avances en la minituarización de los componentes, aumento de la
autonomía de las baterías, mejoras en el control y navegación de las
aeronaves

Mejoras en los algoritmos para la detección de fauna junto al desarrollo
de sensores fotográficos de mayor resolución en un formato más reducido,
permitiendo una menor carga y por tanto un mayor tiempo de vuelo.

Elaborar conclusiones basadas en los resultados obtenidos, destacando
los campos con mayor interés.

A raíz de los resultados obtenidos parece claro que el ámbito de la
conservación se va

Paises con escasos recursos, utilidad del control y vigilancia,
especialmente en areas marinas

\newpage
\singlespacing 
\bibliography{/home/jesus/Documents/CURSOS/drones/RTF/project/master.bib,/home/jesus/Documents/CURSOS/drones/RTF/project/internet.bib}

\end{document}
